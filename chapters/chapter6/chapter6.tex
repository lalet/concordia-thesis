\chapter{Conclusions}\label{conclusion}
This chapter discusses the general conclusions from this study and the future work.

\section{General conclusions}
From this study, we can conclude that the operating system library version updates are affecting the output images from HCP preprocessing pipelines. The analysis of results obtained by processing subjects in different CentOS versions shows that inter-OS differences in these images are visually substantial.

For finding the inter-run differences, analysis was done on results obtained from processing subjects twice in the same condition. No inter-run differences were identified from these results. From this, we can conclude that there are no pseudo-random processes or silent crashes occouring in the HCP preprocessing pipelines.

The likely causes, of the inter-OS differences are, (i) the evolution of math libraries over the time (ii) the pipelines are unstable, i.e, these pipelines amplify the small numerical differences that are created by the differences in the underlying operating system libraries.

There are two ways to tackle this problem. The easy but less preferred solution to this problem would be masking the instabilities. But, the preferred solution is to fix the instabilities in the pipeline, that is, to make the pipelines more stable.

The masking of instabilities can be done by using, (i) single operating system for the processing of subjects, (ii) containerizing the pipelines so that the processing is done on a more controlled environment (researchers can control the updates to libraries and operating system), (iii) Increasing the numerical precision of the pipelines, (iv) Following stricter truncation and rounding standard (IEEE 754), (v) Building static executable by removing the host operating system library dependency.

The above mentioned way is said to be less preferred because it just makes the problem invisible but the instability still remains. If we conduct the study with a change of condition like a newer version of operating system or on an entirely different operating system, the results we obtain would be different.

The preferred solution is to fix the particular functions/scripts in the preprocessing pipelines that are unstable. This would reduce the variance in results considerably and thus, it could make the output more stable and reliable.

\section{Contributions of this thesis}
All the software tools developed for this study are open-source.
%\todo{Be careful here: you used a lot of stuff which is not yours. You should say "All the software tools developed for this project are open-source."}
 Docker images created for this study are hosted on 
 Dockerhub\footnote{\url{https://hub.docker.com/r/bigdatalabteam/hcp-prefreesurfer/}}. 
 Dockerhub status shows that these images has been pulled from the 
 repository more than 8000 times. Repro-tools can be used to analyze 
 the datasets in order to find out the differences in checksum or size 
 of files. It can also quantify the differences with the use of 
 metrics. Different metrics can be configured in Repro-tools 
 (verifyFiles script) to quantify the differences with respect to the 
 file formats. The visualizations we have added to the results section 
 clearly portrays the effect of changing the operating system for 
 conducting the experiments. These images clearly convey the message 
 that operating systems on which the preprocessing takes place has a 
 strong effect on the output image.

\section{Future work}
Though we could identify the files that are having differences and quantify these differences using the metrics, the functions in the pipeline that causes these differences should be identified.
Pipelines should be studied in detail to identify the first instance of difference that gets created and how this difference is propagated through the pipeline.
Since the pipelines are linked to each other (previous pipeline's output becomes next pipeline's input), each pipeline needs to be examined in detail to find out if the pipeline just propagates the error, whether it amplifies them and also if it creates new files with differences as well. 
Study can be extended to other pipelines in the HCP pipelines (e.g., fMRISurface and Diffusion Preprocessing pipeline).
