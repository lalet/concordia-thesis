\chapter{Framework Architecture}
\paragraph{Explanation about the figure}

\section{Framework}
\subsection{Pipeline Containerization}
\subsubsection{Docker images}
\begin{itemize}
  \item Linux CentOS 5.11
  \item Linux CentOS 6.8 
  \item Linux CentOS 7.2.1511
\end{itemize}

\subsubsection{Main Dependencies}
\begin{itemize}
  \item FSL - 5.0.6
  \item FreeSurfer 5.3.0-HCP
\end{itemize}

\subsubsection{HCP Pipelines (v3.19.0)}
\begin{itemize}
  \item FreeSurfer,
  \item PreFreesurfer
  \item PostFreeSurfer
  \item Images are available in Docker Hub
\end{itemize}

\subsection{Pipeline Encapsulation:Boutiques}
\begin{itemize}
  \item Describes the command-line interface of the pipeline
  \item Adds checksum to prevent input data corruption
  \item Checks container image identifier
  \item Records all the software (library versions) and hardware specifications
\end{itemize}


\subsection{Pipeline Deployment:CBRAIN}
\begin{itemize}
  \item Single server to ensure no hardware differences
  \item Able to process multiple subjects at once
  \item CBRAIN HCP plugins available here
\end{itemize}

\subsection{Provenance Capture}
\begin{flushleft}
For every output file, Reprozip captures: 
\end{flushleft}
\begin{itemize}
  \item Process that wrote the file
  \item Command line arguments
  \item Environment variables
  \item Timestamp
  \item Working Directory
\end{itemize}

\subsection{File content evaluation across conditions}
\begin{itemize}
 \item Identify the files that are common to all subjects and all conditions
 \item Check intra-OS variability (e.g., due to pseudo-random processes)
 \item Identifies files with differences (based on md5 checksum)
 \item Associates file types with specific metrics (Normalized RMSE, Text filters, Dice coefficient)
 \item Computes difference matrix for each metric and associates matrix with Reprozip trace
\end{itemize}

\subsubsection{Metrics}
\begin{itemize}
\item NRMSE
\item Checksum
\item Dice
\item Hardware Differences
\end{itemize}


