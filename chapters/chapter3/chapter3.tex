\chapter{Framework Architecture}
\paragraph{Explanation about the figure}

\section{Libraries Used}
\paragraph{Software versions and details}
\subsection{List all the libraries that are used}
\begin{itemize}
 \item bc- calculator for fslreorient2std
 \item curl-devel - cURL is a software package which consists of command line tool and a library for transferring data using URL syntax
 \item epel-release to help pip installation
 \item expat-devel - xml parser library
 \item fontconfig - Fontconfig package contains a library and support programs used for configuring and customizing font access
 \item freetype contains a library which allows applications to properly render TrueType fonts
 \item gettext helps in producing multi-lingual messages
 \item git library for cloining pipelines repository.
 \item libpng - libpng is the official PNG reference library
 \item libSM - is a library of generally useful C abstractions
 \item libstdc++ - a powerful set of reuseable tools in the form of the C++ Standard Library
 \item libXrender - used when compiling programs that use freetype fonts
 \item libXext - Xorg libraries provide library routines that are used within all X Window applications
 \item openssl-devel - for secure communication
 \item perl-ExtUtils-MakeMaker - designed to write a Makefile for an extension module from a Makefile.PL
 \item tar,unzip - to extract .tar.gz and .zip files
 \item wget to get files or status from URL's
 \item zlib-devel - to support FMRIB file IO
 \item numpy - to support FMRIB pipeline execution
\end{itemize}

\section{Algorithms and Metrics}
\paragraph{Algorithms}
\subsection{Describe the ways in which the analysis is done}

\section{Inter Run}
\paragraph{Inter run}

\section{Inter-OS}
\paragraph{Inter OS}

\section{Multiple Run}
\paragraph{Multiple Run}

\section{Repro-tools}
\subsection{Metrics}
\subsubsection{NRMSE}
\subsubsection{Checksum}
\subsubsection{Dice}
\subsubsection{Hardware Differences}
\subsection{Provenance}
\paragraph{Provenance Capture}
