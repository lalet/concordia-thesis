\chapter{A framework for analyzing the reproducibility issues of neuroimaging pipelines}
The Repro-tools \todo{name subject to change } is a framework developed for analyzing the reproducibility issues occurring in the neuroimaging pipelines. Though this framework is developed with various neuroimaging pipelines in focus, in principle, it can be also be used for analyzing data belonging to various other disciplines. The term `condition` with respect to this framework refers to different operating systems on which the data processing takes place. Likewise, the term `subject' with respect to this framework is used to denote the directories that are kept under each condition. 

I contributed a tool, which can compare the files to find differences in their checksum, called verifyFiles\footnote{\url{https://github.com/big-data-lab-team/repro-tools/blob/master/verifyFiles.py}}. The tool can compare and summarize the files, that has differences, due to the operating system on which the processing took place.

The script works under the assumption that, under an ideal scenario, files belonging to each subject is supposed to have equal checksum under different conditions. Repro-tools help in identifying the files having differences by comparing the checksum values of files stored under different conditions. This tool can also check for corruption of files. It can compute the checksum locally and compare it against the recorded checksum to find if the files are corrupted or not.

It can also quantify the differences in image files with the help of metrics like sum of squared distances (SSD), Dice coefficient etc. Additional metrics can be configured which could be used for quantifying differences occurring in different types of files. Repro-tools can also trace the provenance of these files having the differences. It helps us identify the processes that wrote the files having differences. Provenance tracing is done with the use of data captured by Reprozip.

%Refer what is subject also in the above paragraph
%Compute Canada in architecture

\begin{center}
\includegraphics[width=\linewidth]{framework_architecture.png}
\captionof{figure}{Framework Architecture}
\label{fig:framework_architecture}
\end{center}

Figure \ref{fig:framework_architecture} illustrates the overall workflow. The Docker image creation is automated. For each commit, made to the Dockerfile stored in Github\footnote{\url{https://github.com/big-data-lab-team/Dockerfiles-HCP-PreFreesurfer}}, a new build is triggered in Dockerhub\footnote{\url{https://hub.docker.com/r/bigdatalabteam/hcp-prefreesurfer/}}. These images are then used by CBRAIN for processing the subjects. Boutiques helps in deploying these containers to CBRAIN platform. CBRAIN can harness the power of computational clusters as well. The subjects processed under different conditions are then analyzed using the Repro-tools.

\subsection{Docker images}
Repro-tools framework uses Docker containers for reproducibility studies. A Docker container is the running instance of a Docker image. Dockerfiles are used for specifiying the operating system and the libraries needed for creating a Docker image. With the help of automated image creation feature in Dockerhub, whenever a commit is made to the Dockerfile, a new image build gets triggered. Thus, the automated build makes sure that the changes are always reflected in the Docker images. 

Repro-tools, before the beginning of processing a subject, makes a check to ensure that the image that is getting used is the latest one. This check makes sure that the processing is done with the up-to-date changes in the Docker images.

\section{Pipeline Encapsulation}
%Add Why wrapper and advantages
I created a wrapper script\footnote{\url{https://github.com/big-data-lab-team/Dockerfiles-HCP-PreFreesurfer/blob/master/PreFreeSurfer-DockerFiles/command-line-script.sh}} on top of the pipeline to have the features which are listed below,
\begin{itemize}
  \item Compute the checksum of the files in each subject before and after the execution
  \item Create execution directory and copy the subject to prevent corrupting the input data
  \item Record all the software (library versions) present in the container and hardware specifications of the workstation
  \item Ability to trace the execution using Reprozip (optional)
\end{itemize}

The checksums are computed before and after execution so that corruption check can be done with the use of recorded checksums. After the execution directory creation, the subject folder is copied into the execution directory and thus, it prevents the corruption of the input data. Another feature, recording of the software library versions and hardware specifications are done in order to make sure that the only factor that changes in these experiments is the operating system version. The optional Reprozip tracing feature is used to record the details of the pipeline while the processing takes place so that it can be used as a referrence for provenance tracing.

\section{Pipeline Deployment}
The Docker container containing the pipelines along with the Boutiques descriptor was deployed on a server setup as part of our study. With the help of CBRAIN along with the containers and the descriptors, we were able to process the subjects. Single server was used in order to prevent differences occurring in the files due to differences arising from the hardware architecture. Boutiques descriptors used for deploying the pipelines on CBRAIN is available at \cite{HCP_descriptors}.

%\begin{itemize}
 % \item Single server to ensure no hardware differences
  %\item Able to process multiple subjects at once
  %\item CBRAIN HCP plugins available here
%\end{itemize}

\section{File comparisons across conditions}
%Checksum
Repro-tools framework can compare the files across different conditions based on the checksum. We have used MD5\footnote{\url{https://tools.ietf.org/html/rfc1321}} algorithm for calculating the checksum of files. The output of a MD5 algorithm is a 128-bit ``fingerprint" or ``message digest"~\cite{md5}. Though MD5 is not completely secure against collision attack, Repro-tools focus more on the data integity than security and how fast the algorithm is able to create the checksum. These qualities make MD5 the best choice for checksum generation in Repro-tools.

Two types of differences can occur in the subjects due to the differences in the operating systems. One is inter-OS difference which occur due the the operating system library updates and the other type, intra-OS differences occur due to pseudo-random processes used in the pipelines. An example of such a function is, a random number generator that would get initialized using a seed state. Repro-tools can be used to identify both kind of differences.

The files that are common to all the subjects only are taken into consideration for comparison. The first step is identification of files with differences in their checksums. This is identified using the checksums that are recorded after the processing. Intra-OS differences are identified using the run-number added as the suffix for the conditions. For example, the two batches of subjects processed under the same condition (CentOS6) are stored as run-1 and run-2. The files belonging to the subjects stored under the above mentioned conditions are treated as intra-OS runs.

For the files that are identified to have differences, different kind of metrics are used base on the file type to quantify the differences. Normalized root mean square error, Dice Coefficient, Text filter etc. are the various metrics used for quantifying the differences. Section \ref{sec:num1} explains the metrics in detail. 

These metric values help us understand how big or small the differences are. Apart from quantifying the differences using type specific metrics, Repro-tools can also be used to trace the provenance of these differences. It is able to identify all the processes and associated parameters that wrote the files having differences. These details about various processes is helpful in debugging the pipelines. This information helps in recreating the processing step by step and also to identify the processes that creates the differences.

%Add algorithm

\section{Provenance Capture}
Repro-tools traces the provenance of differences using the data captured by Reprozip. Reprozip records the data in an SQLite\footnote{\url{https://www.sqlite.org/}} database. Repro-tools framework queries this database to find out the provenance information about the files that has a difference in their checksum. Figure \ref{fig:provenance-query} contains the query used for finding the provenance information.\\

\begin{tcolorbox}[colback=black!5!white,colframe=black!75!black]
SELECT DISTINCT executed\_files.name, executed\_files.argv, executed\_files.envp, executed\_files.timestamp, executed\_files.workingdir from executed\_files INNER JOIN opened\_files where opened\_files.process = executed\_files.process and opened\_files.name like ? and opened\_files.mode=2 and opened\_files.is\_directory=0',('\%/'+file\_name,)
\end{tcolorbox}
\captionof{figure}{Query for provenance information}
\label{fig:provenance-query}

Query makes use of details about the 1) processes that wrote the file; 2) command line arguments used by those processes; 3) Environment variables used; 4) timestamp; 5) working directory. By the help of query shown in \ref{fig:provenance-query}, we can identify the processes in the pipeline that create these differences.

\section{Metrics} \label{sec:num1}
The metrics that are used to quantify the differences are described below.

\subsection{Normalized Root Mean Square Error (NRMSE)}
As defined in \cite{khosrow2017handbook}, ``The Root Mean Square Deviation (RMSD) or root-mean-square error (RMSE) is a frequently used measure of the difference between values predicted by a model or an estimator and the values actually observed". Normalizing the RMSD value makes it easier to compare between models or datasets.

Example from \cite{NRMSE}, the RMSE of predicted values ${\displaystyle {\hat {y}}_{t}}$  for times t of a regression's dependent variable ${\displaystyle y_{t}}$ is computed for n different predictions as the square root of the mean of the squares of the deviations:\\

\begin{center}
  \begin{equation}
     RMSE = {\sqrt {\frac{1} {n}{\sum\limits_{t = 1}^n {(\hat{y}_{t} - {y}_{t} } })^{2} } }
  \end{equation}
\end{center}

\begin{center}
  \begin{equation}
    NRMSE = {\frac{RMSE} {y_{max} - y_{min}}}
  \end{equation}
\end{center}

\subsection{Dice Similarity Coefficient}
Dice Similarity Coefficient can be used as a statistical validation metric for measuring the reproducibility of magnetic resonance images~\cite{Zou2004}.
\begin{center}
  \begin{equation}
     J(A,B) = {{|A \cap B|}\over{|A \cup B|}} = {{|A \cap B|}\over{|A| + |B| - |A \cap B|}}
  \end{equation}
\end{center}
\begin{itemize}
\item NRMSE
\item Dice
\end{itemize}
