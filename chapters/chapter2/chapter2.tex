\chapter{Related Work}

\section{Reproducibility across operating systems}
\begin{itemize}
 \item Reproducibility of neuroimaging analyses across operating systems, T. Glatard, L. B. Lewis, R. Ferreira da Silva, R. Adalat, N. Beck, C. Lepage, P. Rioux, M. Rousseau, T. Sherif, E. Deelman, Najmeh Khalili-Mahani, Alan C. Evans, Frontiers in Neuroinformatics, vol. 9, no. 12, 2015.
 \item The Effects of FreeSurfer Version, Workstation Type, and Macintosh Operating System Version on Anatomical Volume and Cortical Thickness Measurements, Ed H. B. M. Gronenschild, Petra Habets, Heidi I. L. Jacobs, Ron Mengelers, Nico Rozendaal, Jim van Os, and Machteld Marcelis.
 \item Reprozip: Computational reproducibility with ease., Chirigati, Fernando, et al, Proceedings of the 2016 International Conference on Management of Data. ACM, 2016
 \item Software architectures to integrate workflow engines in science gateways, Tristan Glatard, Marc-Étienne Rousseau, Sorina Camarasu-Pop, Reza Adalat, Natacha Beck, Samir Das, Rafael Ferreira da Silva, Najmeh Khalili-Mahani, Vladimir Korkhov, Pierre-Olivier Quirion, Pierre Rioux, Sı́lvia D. Olabarriaga, Pierre Bellec, Alan C. Evans, Future Generation Computer Systems, Volume 75, October 2017, Pages 239-255.
\end{itemize}

\section{Technologies for reproducible computing}
A key component of the scientific method is the ability to revisit and replicate previous results.Registering information about an experiment allows scientists to interpret and
understand results, as well as verifying that the experiment was performed according to acceptable procedures. To carry out a computer science experiment we make use of the entiretyof the computer which is used for running the experiment. Repeating a computer science experiment doesn't require a scientist to rewrite the code from scratch but getting access to the code and resources suffice. Thus in principle a well documented code should be easily reproducible, but that is not the case. Computing environments change rapidly in terms of hardware as well as software. So just documenting the code base won't be adequate enough for easily reproducing the experiments.

Thus the rapid changes occuring in computing environments are complex and might cause differences in outputs eventhough the correct version of the code was ran on the exact same computing environment.To address this problem virtual machines could be used. But the overheads in terms of performance and management(creating, storing and transferring them) can be high and, in some fields of Computer Science such as Computer Systems research it can't be accounted easily.\cite{7092948}
 
\section{Containers}
\subsection{What are containers?}
\subsection{How containers help reproducibility?}
\subsection{What are its limitations?}
\subsection{Include the problems listed in link https://goo.gl/zodrok}
\subsection{Softwares for reproducible computing}
\subsubsection{Docker}
\subsubsection{Singularity}
\subsubsection{Differences and similarities between Docker and Singularity}

\section{Web platforms to run containers}
\subsection{Why do we need these platforms?}
\subsection{Examples of web computing platforms}
\subsubsection{CBRAIN}
\subsubsection{Amazon}

\section{Boutiques}
\subsection{What is Boutiques}
\subsection{How it helps us to run the containers?}

\section{Interposition Techniques}
\subsection{System call interception}
\subsection{Library call interception}
\subsection{Reprozip tool}

\section{NeuroImaging Pipelines}
\begin{itemize}
 \item FSL
 \item FreeSurfer
 \item HCP Pipelines
\end{itemize}


