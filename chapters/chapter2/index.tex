\chapter{Materials and Methods}

\section{HCP Data}
\paragraph{Details about the subjects}

\section{Operating System and Applications}
\paragraph{Operating System}
\subsection{CentOS Intro and description}

\section{Minimal preprocessing pipelines}
\paragraph{Minimal preprocessing pipelines Overview}
\subsection{What is preprocessing}
\subsection{PrefreeSurfer}
\subsection{FreeSurfer}
\subsection{PostFreeSurfer}
\subsection{fMRIVolume}
\subsection{fMRISurface}
\subsection{Diffusion Preprocessing}

\section{FSL installation}
\paragraph{FSL}
\subsection{FSL}

\section{Hardware}
\paragraph{Hardware}
\subsection{Specifications}

\section{Softwares and Libraries}
\paragraph{Software and Libraries}
\subsection{List all the libraries that are used}
\begin{itemize}
 \item bc- calculator for fslreorient2std
 \item curl-devel - cURL is a software package which consists of command line tool and a library for transferring data using URL syntax
 \item epel-release to help pip installation
 \item expat-devel - xml parser library
\item fontconfig - Fontconfig package contains a library and support programs used for configuring and customizing font access
 \item freetype contains a library which allows applications to properly render TrueType fonts
 \item gettext helps in producing multi-lingual messages
\item git library for cloining pipelines repository.
\item libpng - libpng is the official PNG reference library
\item libSM - is a library of generally useful C abstractions
\item libstdc++ - a powerful set of reuseable tools in the form of the C++ Standard Library
\item libXrender - used when compiling programs that use freetype fonts
\item libXext - Xorg libraries provide library routines that are used within all X Window applications
\item openssl-devel - for secure communication
\item perl-ExtUtils-MakeMaker - designed to write a Makefile for an extension module from a Makefile.PL
\item tar,unzip - to extract .tar.gz and .zip files
\item wget to get files or status from URL's
\item zlib-devel - to support FMRIB file IO
\item numpy - to support FMRIB pipeline execution
\end{itemize}

\subsection{Docker}

\subsection{Reprozip}

\subsection{Nibabel}

\section{Algorithms and Metrics}
\paragraph{Algorithms}
\subsection{Describe the ways in which the analysis is done}
\subsection{NRMSE}
\subsection{Checksum}
\subsection{Dice}
\subsection{Hardware Differences}



\section{Pipeline Analysis}
\paragraph{Pipeline Analysis}

\section{Inter run experiment}
\paragraph{Same  OS , separate runs, same subjects}

\section{Inter OS experiment}
\paragraph{Different OS, same subjects}


\section{Results}
\paragraph{Results}

\subsection{Inter Run results}
\paragraph{Same  OS , separate runs, same subjects}

\subsection{Inter OS results}
\paragraph{Different OS, same subjects}

\section{Discussion}
\paragraph{Discussion}

\section{Conclusion}
\paragraph{Conclusion}




