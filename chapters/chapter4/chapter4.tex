\chapter{Application to HCP pre-processing pipelines}

\subsubsection{HCP Pipelines (v3.19.0)}
%Intro to HCP preprocessing 
HCP preprocessing pipelines are designed to minimize the amount of information removed from the HCP data. Figure \ref{fig:hcp_pipelines_overview} illustrates the HCP Preprocessing Pipelines overview. HCP pipelines consists of three structural pipelines (PreFreeSurfer, FreeSurfer and PostFreeSurfer), two functional pipelines (fMRIVolume and fMRISurface) and a Diffusion PreProcessing pipeline~\cite{Gla13}. According to~\cite{Gla13}, the overall goals of HCP Preprocessing pipelines are, ``1) to remove spatial artifacts and distortions; 2) to generate cortical surfaces, segmentations, and myelin maps; 3) to make the data easily viewable in the Connectome Workbench visualization software; 4) to generate precise within-subject cross-modal registrations; 5) to handle surface and volume cross-subject registrations to standard volume and surface spaces; and 6) to make the data available in the Connectivity Informatics Technology Initiative (CIFTI) format in a standard ``grayordinate" space". Grayordinate represents the gray matter in the brain using a surface vertex or a volume voxel~\cite{Grayordinate}. The minimal preprocessed subjects are available in standard format which makes it easier to compare it with subjects from other studies. This standardization makes it easier for researchers to report their findings and replicate the studies.

\begin{center}
   \includegraphics[width=\linewidth]{hcp_preprocessing_overview}
   \captionof{figure}{HCP Preprocessing Pipelines Overview}
   \label{fig:hcp_pipelines_overview}
   \caption*{Extracted from \cite{Gla13}}
\end{center}

Out of the six pipelines illustrated in Figure \ref{fig:hcp_pipelines_overview}, we used three structural pipelines and one functional pipeline for study. The pipelines are listed below, 
\begin{itemize}
  \item PreFreeSurfer
  \item FreesSurfer
  \item PostFreeSurfer
  \item fMRIVolume
\end{itemize}

The HCP pipeline is open sourced, and we used the version 3.19.0\footnote{\url{https://github.com/Washington-University/Pipelines/releases/tag/v3.19.0}} for our study. The details of pipeines are listed below. 

\subsubsection{PreFreeSurfer}
Main goals of, PreFreeSurfer, according to~\cite{Gla13} are, 1) to produce an undistorted ``native" structural volume space for each subject; 2) align the T1W and T2W images; 3) perform bias field correction; 4) register the subject's native structural volume space to MNI space. The workflow of PreFreeSurfer is illustrated in Figure \ref{fig:prefreesurfer_overview}.

\begin{center}
  \includegraphics[width=\linewidth]{PreFreeSurfer_Architecture}
  \captionof{figure}{PreFreeSurfer Overview}
  \label{fig:prefreesurfer_overview}
  \caption*{Extracted from \cite{Gla13}}
\end{center}

PreFreeSurfer starts with correcting the distortions caused by magnetic resonance gradient nonlinearity~\cite{Gla13}. According to~\cite{Zou2004}, ``Gradient nonlinearity is a static characteristic of the gradient coil system known to system engineers and universally utilized for correction of geometric distortions for routine MRI scans". All images used in structural processing (T1w, T2w, the field map and phase) must be corrected from gradient nonlinearity distortion. The correction is done with a customized version of gradient\_nonlin\_unwarp package available in FreeSurfer. As per the definition given by~\cite{t1w_t2w}, T1-weighted images provides good contrast between gray matter and white matter in brain while cerebrospinal fluid is void of signal (black color) and T2-weighted images provides good contrast between cerebrospinal fluid and brain tissue. Figures~\ref{fig:T1w} and~\ref{fig:T2w} illustrates T1-weighted and T2-weighted images.\\

\begin{center}
  \includegraphics[width=\linewidth]{T1_v3}
  \caption{T1-weighted image}
  \label{fig:T1w}
  \caption*{Extracted from \cite{t1w_t2w}}
\end{center}

\begin{center}
  \includegraphics[width=\linewidth]{T2_v1}
  \caption{T2-weighted image}
  \label{fig:T2w}
  \caption*{Extracted from \cite{t1w_t2w}}
\end{center}

Next step is aligning the T1w and T2w images, using FSL's FLIRT. Subsequent step is aligning the average T1w and T2w images to the MNI space template. The main purpose behind this step is to have the same orientation as the template for the ease of visualization.

After this ACPC alignment step, a robust initial brain extraction step is performed using FSL's linear (FLIRT) and non-linear (FNIRT) registration. Removing the readout distortion is the last step in creating the subject's undistorted native volume space. Readout distortion is caused when the digital data is retrieved from an electronic medium and is is displayed in a human understandable format.

For correcting the gradient nonlinearity distortion, the images are converted into a field map using the fsl\_prepare\_fieldmap\_script. According to \cite{field_map}, ``A field map is an image of the intensity of the magnetic field across space". Mean magnitude and field map images are corrected for gradient nonlinearity distortion. The field map is then transformed according to these registrations and used to unwarp the T1w and T2w images, removing the differential readout distortion present in them. The readout-distortion-corrected T1w image, which now has all spatial distortions removed from it, represents the native volume space for each subject. The undistorted T2w image is cross-modally registered to the T1w image using FLIRT's boundary based registration (BBR) cost function. Boundary based registration focus on aligning intensity gradients across tissue boundaries. Once T1w and T2w images are in the same space , intensity inhomogeneity correction is applied on these images. After the bias field correction , T1w image is registered to MNI space after a FLIRT registration followed by FNIRT nonlinear registration.

The output of PreFreeSurfer pipeline are organized into a folder called T1w that contains native volume space images and a second folder (MNINonLinear) that contains MNI space images~\cite{Gla13}.

\subsubsection{FreeSurfer}
HCP pipelines uses FreeSurfer version 5.2 with a lot of enhancements made particulariy focusing HCP data. The main goals, according to~\cite{Gla13} are, 1) improve the robustness of brain extraction; 2) fine tune T2w to T1w registration; 3) accurately place the white and pial surface with high resolution data; 4) perform FreeSurfer's standard folding-based surface registration to their surface atlas (fsaverage). The workflow of FreeSurfer is illustrated in Figure~\ref{fig:freesurfer_overview}.

\begin{center}
  \includegraphics[width=\linewidth]{FreeSurfer_Architecture}
  \captionof{figure}{FreeSurfer Overview}
  \label{fig:freesurfer_overview}
  \caption*{Extracted from \cite{Gla13}}
\end{center}

After the PreFreeSurfer processing, FreeSurfer's recon-all pipeline is used for processing the output from the PreFreeSurfer pipeline. Recon-all pipeline is not able to process structural images having high resolution. So HCP data is downsampled to meet the requirements of the recon-all pipeline.

The T1w registration in the FreeSurfer is aided using the initial brain mask generated in PreFreeSurfer. The important steps that takes place before the recon-all process includes, automated segmentation of the T1w volume and tessellation and topology correction of the initial white matter surface. The resultant white matter surface is generated using a segmentation of the 1 mm downsampled T1w image.

The white matter surface placement is done using the original high resolution T1w images. The FreeSurfer volume and surface files brought into the .7 mm native volume space, the high resolution T1w volume is intensity normalized and the white matter surface position is adjusted based on intensity gradients in the .7 mm T1w image.

The T2w to T1w registration is fine-tuned using FreeSurfer's BBRegister algorithm. The corrected white matter surfaces are then brought back into FreeSurfer space and recon-all processing continues.

Next step is the generation of the Pial surfaces. The initial pial surfaces are generated from the high-resolution PreFreeSurfer bias-corrected T1w image.
The T2w images are used to remove any dura and blood vessels present in the image.

\subsubsection{PostFreeSurfer}
Main goals of, PostFreeSurfer, according to~\cite{Gla13} are, 1) produces all of the NIFTI volume and GIFTI surface files necessary for viewing the data in Connectome Workbench,2) application of surface registration; According to \cite{DBLP:journals/corr/HrgeticP13}, ''task of the surface registration algorithms is to determine the corresponding surface parts in the pair of observed clouds of 3D points, and on that basis to determine the spatial translation and rotation between the two views"; 3) downsampling registered surfaces for connectivity analyses; 4) creating the final brain mask, and creating myelin maps. Figure \ref{fig:postfreesurfer_overview} illustrates the workflow of PostFreeSurfer.\\

\begin{center}
  \includegraphics[width=\linewidth]{PostFreeSurfer_Overview}
  \captionof{figure}{PostFreeSurfer Overview}
  \label{fig:postfreesurfer_overview}
  \caption*{Extracted from \cite{Gla13}}
\end{center}

\subsubsection{fMRIVolume}
Main goals of, fMRIVolume, according to~\cite{Gla13} are, 1) remove spatial distortions; 2) realignment of volumes to adjust subject motion; 3) registration of fMRI data to structural volume; 4) normalizaton of the 4D image; 5) mask the data with final brain mask. Figure \ref{fig:fMRIVolume_overview} illustrates the workflow of fMRIVolume pipeline.\\

\begin{center}
  \includegraphics[width=\linewidth]{fMRI_Volume}
  \captionof{figure}{fMRI Volume Overview}
  \label{fig:fMRIVolume_overview}
  \caption*{Extracted from \cite{Gla13}}
\end{center}

\section{Data}
\paragraph{HCP Data Acquisition}
The data was acquired from Human Connectome Project\footnote{\url{https://db.humanconnectome.org}}. We used the subjects 101006, 101017, 101410, 104820 and 105216. All these subjects were scanned using the HCP3T type of scanner. Figures \ref{fig:subject_details} and \ref{fig:subject_scan_details} gives more insight about the HCP subjects that were used for our study. 

\begin{center}
\includegraphics[width=\linewidth]{subjects_details.png}
\captionof{figure}{Subject Details}
\label{fig:subject_details}
\caption*{Extracted from \cite{DBConnectomeSite}}
\end{center}

\begin{center}
\includegraphics[width=\linewidth]{subject_scan_details.png}
\captionof{figure}{Subject Scan Details}
\label{fig:subject_scan_details}
\caption*{Extracted from \cite{DBConnectomeSite}}
\end{center}

\section{Processing}
The data was processed in the order PreFreeSurfer, FreeSurfer, PostFreeSurfer and fMRIVolume.
Two runs were made on CentOS6 and CentOS7. Tables \ref{tab:prefreesurfer_processing_centos7} and \ref{tab:prefreesurfer_processing_centos6} shows the details about processing subjects on CentOS7 and CentOS6 respectively.

\subsection{PreFreeSurfer}
\begin{center}
\tabulinesep=1.2mm
\begin{tabu} to \textwidth { | X[l] | X[l] | X[l] | }
  \hline
  Subject & Run no. & Time \\
  \hline
  101006 & RUN-1  & 0h and 56min \\
  \hline
  101107 & RUN-1  & 1h and 16min \\
  \hline
  101410 & RUN-1  & 1h and 27min \\
  \hline
  104820 & RUN-1  & 1h and 16min \\
  \hline
  105216 & RUN-1  & 1h and 36min \\
  \hline
  101006 & RUN-2  & 1h and 28min \\
  \hline
  101107 & RUN-2  & 1h and 26min \\
  \hline
  101410 & RUN-2  & 1h and 18min \\
  \hline
  104820 & RUN-2  & 1h and 10min \\
  \hline
  105216 & RUN-2  & 1h and 29min \\
  \hline
\end{tabu}
\captionof{table}{PreFreeSurfer Processing Details on CentOS7}
  \label{tab:prefreesurfer_processing_centos7}
\end{center}

\begin{center}
\tabulinesep=1.2mm
\begin{tabu} to \textwidth { | X[l] | X[l] | X[l] | }
  \hline
  Subject & Run no. & Time \\
  \hline
  101006 & RUN-1  & 1h and 17min \\
  \hline
  101107 & RUN-1  & 1h and 19min \\
  \hline
  101410 & RUN-1  & 1h and 09min \\
  \hline
  104820 & RUN-1  & 1h and 00min \\
  \hline
  105216 & RUN-1  & 1h and 17min \\
  \hline
  101006 & RUN-2  & 1h and 18min \\
  \hline
  101107 & RUN-2  & 1h and 16min \\
  \hline
  101410 & RUN-2  & 1h and 12min \\
  \hline
  104820 & RUN-2  & 0h and 58 min \\
  \hline
  105216 & RUN-2  & 1h and 31min \\
  \hline
\end{tabu}
\captionof{table}{PreFreeSurfer Processing Details on CentOS6}
  \label{tab:prefreesurfer_processing_centos6}
\end{center}
%Prefreesurfer average timings to be calculated
\subsection{FreeSurfer}

\begin{center}
\tabulinesep=1.2mm
\begin{tabu} to \textwidth { | X[l] | X[l] | X[l] | }
  \hline
  Subject & Run no. & Time \\
  \hline
  101006 & RUN-1  & 07h and 17min \\
  \hline
  101107 & RUN-1  & 06h and 56min \\
  \hline
  101410 & RUN-1  & 08h and 32min \\
  \hline
  104820 & RUN-1  & 06h and 20min \\
  \hline
  105216 & RUN-1  & 08h and 31min \\
  \hline
  101006 & RUN-2  & 06h and 52min \\
  \hline
  101107 & RUN-2  & 07h and 01min \\
  \hline
  101410 & RUN-2  & 09h and 14min \\
  \hline
  104820 & RUN-2  & 07h and 34min \\
  \hline
  105216 & RUN-2  & 07h and 40min \\
  \hline
\end{tabu}
\captionof{table}{FreeSurfer Processing Details on CentOS7}
\label{tab:freesurfer_processing_centos7}
\end{center}

\begin{center}
\tabulinesep=1.2mm
\begin{tabu} to \textwidth { | X[l] | X[l] | X[l] | }
  \hline
  Subject & Run no. & Time \\
  \hline
  101006 & RUN-1 & 06h and 15min \\
  \hline
  101107 & RUN-1 & 06h and 46min \\
  \hline
  101410 & RUN-1 & 08h and 11min \\
  \hline
  104820 & RUN-1 & 06h and 13min \\
  \hline
  105216 & RUN-1 & 10h and 03min \\
  \hline
  101006 & RUN-2 & 05h and 46min \\
  \hline
  101107 & RUN-2 & 07h and 18min \\
  \hline
  101410 & RUN-2 & 08h and 06min \\
  \hline
  104820 & RUN-2 & 06h and 21min \\
  \hline
  105216 & RUN-2 & 09h and 12min \\
  \hline
\end{tabu}
\captionof{table}{FreeSurfer Processing Details on CentOS6}
\label{tab:freesurfer_processing_centos6}
\end{center}

\subsection{PostFreeSurfer}

\begin{center}
\tabulinesep=1.2mm
\begin{tabu} to \textwidth { | X[l] | X[l] | X[l] | } 
  \hline
  Subject & Run no. & Time \\
  \hline
  101006 & RUN-1 & 24min \\
  \hline
  101107 & RUN-1 & 20min \\
  \hline
  101410 & RUN-1 & 23min \\
  \hline
  104820 & RUN-1 & 19min \\
  \hline
  105216 & RUN-1 & 29min \\
  \hline
  101006 & RUN-2 & 21min \\
  \hline
  101107 & RUN-2 & 24min \\
  \hline
  101410 & RUN-2 & 22min \\
  \hline
  104820 & RUN-2 & 21min \\
  \hline
  105216 & RUN-2 & Add item \\
  \hline
\end{tabu}
\captionof{table}{PostFreeSurfer Processing Details on CentOS6}
\label{tab:postfreesurfer_processing_centos6}
\end{center}

\begin{center}
\tabulinesep=1.2mm
\begin{tabu} to \textwidth { | X[l] | X[l] | X[l] | }
  \hline
  Subject & Run no. & Time \\
  \hline
  101006 & RUN-1 & 31min \\
  \hline
  101107 & RUN-1 & 31min \\
  \hline
  101410 & RUN-1 & 28min \\
  \hline
  104820 & RUN-1 & 29min \\
  \hline
  105216 & RUN-1 & 32min \\
  \hline
  101006 & RUN-2 & 28min \\
  \hline
  101107 & RUN-2 & 22min \\
  \hline
  101410 & RUN-2 & 28min \\
  \hline
  104820 & RUN-2 & 26min \\
  \hline
  105216 & RUN-2 & 28min \\
  \hline
\end{tabu}
\captionof{table}{PostFreeSurfer Processing Details on CentOS7}
\label{tab:postfreesurfer_processing_centos7}
\end{center}

\subsection{fMRIVolume}

\begin{center}
\tabulinesep=1.2mm
\begin{tabu} to \textwidth { | X[l] | X[l] | X[l] | }
  \hline
  Subject & Run no. & Time \\
  \hline
  101006 & RUN-1 & 1 day 5 hours 17 min \\
  \hline
  101107 & RUN-1 & 1 day 5 hours 07 min\\
  \hline
  101410 & RUN-1 & 18 hours 20 min \\
  \hline
  104820 & RUN-1 & 17 hours 53 min \\
  \hline
  105216 & RUN-1 & add details \\
  \hline
  101006 & RUN-2 & 1 day 5 hours 07 min \\
  \hline
  101107 & RUN-2 & 1 day 4 hours 49 min \\
  \hline
  101410 & RUN-2 & 18 hours 26 min \\
  \hline
  104820 & RUN-2 & 17 hours 10 min \\
  \hline
  105216 & RUN-2 & add details \\
  \hline
\end{tabu}
\captionof{table}{fMRIVolume Processing Details on CentOS6}
\label{tab:fMRIVolume_processing_centos6}
\end{center}

\begin{center}
\tabulinesep=1.2mm
\begin{tabu} to \textwidth { | X[l] | X[l] | X[l] | }
  \hline
  Subject & Run no. & Time \\
  \hline
  101006 & RUN-1 & 1 day 4 hours 18 min \\
  \hline
  101107 & RUN-1 & 1 day 4 hours 22 min\\
  \hline
  101410 & RUN-1 & 17 hours 46 min \\
  \hline
  104820 & RUN-1 & 17 hours 08 min \\
  \hline
  105216 & RUN-1 & 1 day 5 hours 7 min \\
  \hline
  101006 & RUN-2 & Add details  \\
  \hline
  101107 & RUN-2 & Add details  \\
  \hline
  101410 & RUN-2 & Add details  \\
  \hline
  104820 & RUN-2 & 1 day 4 hours 15 min \\
  \hline
  105216 & RUN-2 & 1 day 5 hours 27 min\\
  \hline
\end{tabu}
\captionof{table}{fMRIVolume Processing Details on CentOS7}
\label{tab:fMRIVolume_processing_centos7}
\end{center}

%Add average time of all the processes

\paragraph{File comparision across conditions using metrics}
\begin{itemize}
  \item NRMSE
  \item Dice
  \item Checksum
  \item Hardware
\end{itemize}
