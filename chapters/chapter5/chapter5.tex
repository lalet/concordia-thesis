\chapter{Results}
This chapter discusses the results we obtained by processing data on HCP pipelines. The differences we observed and the quantified metric values are discussed in detail for each pipeline. Section \ref{sec:Prefreesurfer} discusses PrefreeSurfer results, followed by section \ref{sec:Freesurfer} on FreeSurfer results, section \ref{sec:Postfreesurfer} on PostFreeSurfer results and last section \ref{sec:fMRI} discusses fMRIVolume pipeline results.

\section{PreFreeSurfer} \label{sec:Prefreesurfer}
The subjects were processed on CentOS6 and CentOS7 using the PreFreeSurfer pipeline. Among the 92 NIfTI imaging files common to all 5 subjects, 76 (.nii.gz) files differed between CentOS6 and CentOS7. Figure~\ref{fig:prefreesurfer_metric_values} shows the Dice coefficient value and NRMSE values of the NIfTI files that were found to have differences in between operating systems. The mean, median and standard deviation NRMSE and Dice coefficient values are given in Table~\ref{tab:PreFreeSurfer_Metic_Values}.

\hfill \break
\begin{center}
\begin{tabular}{|l|l|l|}
\hline
\textbf{Item}      & \textbf{Nrmse} & \textbf{Dice} \\ \hline
Mean               & 0.0069   & 0.2295   \\ \hline
Median             & 0.0037    & 0.0216    \\ \hline
Standard Deviation & 0.0139    & 0.3371   \\ \hline
\end{tabular}
\captionof{table}{NRMSE \& DICE values for PreFreeSurfer processing on CentOS6 and CentOS7}
\label{tab:PreFreeSurfer_Metic_Values}
\end{center}
\hfill \break

Table \ref{tab:PreFreeSurfer_Metic_Values} contains the average mean, median and standard deviation we calculated for on the metircs, NRMSE value and Dice coefficient. The NRMSE value from the PreFreeSurfer pipeline has an average of 0.0069 and the Dice Coefficient average is 0.229. Dice Coefficient shows that out of the results we obtained from PreFreeSurfer processing on two conditions, there was only 22\% similarity while taking the average of dice coeffiecient of every file that we found to have a checksum difference.

\vskip 0.2in
Figure \ref{fig:prefreesurfer_metric_values} contains the matrices showing the plotted values of NRMSE (left) and Dice Coefficient (right). Each line in the matrix corresponds to a file common to all the subjects. Each column represents a subject and each row represents the same file in 5 different subjects. The NRMSE value appears brighter if the values are high and it appears darker if the values are low. The Dice coefficient values appears darker if the images are dissimiliar and appears white if the images are similiar.

\hfill \break
\begin{center}
\includegraphics[width=.5\linewidth]{prefreesurfer_nrmse.png}%
\includegraphics[width=.5\linewidth]{prefreesurfer_dice.png}
\captionof{figure}{PreFreeSurfer metric values}
\caption*{(i) NRMSE (left) (ii)Dice coefficient (right)}
\label{fig:prefreesurfer_metric_values}
\end{center}
\hfill \break

Figure~\ref{fig:prefreesurfer_t1w_acpc} illustrates the binarized differences in the ``T1wmulT2w\_brain\_norm\_s5.nii.gz file". The file was taken from subject ``101410". File has an NRMSE value of 0.021 and the Dice coefficient value is 0.007. The thick blue shaded region shows the difference in between the images in two conditions (CentOS6 and CentOS7).

\hfill \break
\begin{center}
\includegraphics[width=.9\linewidth]{t1wmul1.png}
  \captionof{figure}{PreFreeSurfer T1wmulT2w brain normalization file (Blue shaded region shows the binarized differences and the green shaded region around the border is part of file from CentOS6). \href{https://drive.google.com/file/d/1RYlwkn6Aqul9DeljxnC78FpJhoVQBElP/view?usp=sharing}{Link}}
\caption*{(Subject: 101410; Filename: T1wmulT2w\_brain\_norm\_s5.nii.gz; Dice coeff.: 0.007 ; NRMSE: 0.021)}
\label{fig:prefreesurfer_t1w_acpc}
\end{center}
\hfill \break

Figure~\ref{fig:prefreesurfer_t2w_acpc} illustrates the differences in the ``T2w\_acpc\_nii.gz". The file was taken from subject ``105216". File has an NRMSE value of 0.0065 and the Dice coefficient value is 0.6*10\textsuperscript{-6}. The square pixels in the image shows the differences in between two conditions (CentOS6 and CentOS7).

\hfill \break
\begin{center}
\includegraphics[width=.3\linewidth]{checkerboard_T2w_1.png}%
\includegraphics[width=.3\linewidth]{checkerboard_T2w_2.png}%
\includegraphics[width=.3\linewidth]{checkerboard_T2w_3.png}
\captionof{figure}{PreFreeSurfer T2w ACPC file (Pixellated voxels shows the differences).\href{https://drive.google.com/file/d/1NrNl7POyCS_SZm3an00wOLUnkmLJ_ngo/view?usp=sharing}{Link}}
\caption*{(Subject: 105216; Filename: T2w\_acpc\_nii.gz; Dice coeff.: 0.6*10\textsuperscript{-6} ; NRMSE: 0.0065)}
\label{fig:prefreesurfer_t2w_acpc}
\end{center}
\vskip 0.2in
The visualization of the differences in T2w ACPC file given in the link above, shows that the images from two conditions are totally out of alignment. This differnence is the result of PreFreeSurfer processing (steps 4-6, Figure \ref{fig:prefreesurfer_overview}). The non-linear registration done using FNIRT creates the difference and this registration error is propagated as a substantial difference in the resulting brain mask.
\vskip 0.2in
Figure~\ref{fig:prefreesurfer_std_file} illustrates the differences in the ``standard2str.nii.gz". The file was taken from subject ``105216". File has an NRMSE value of 0.036 and the Dice coefficient value is 0.2*10\textsuperscript{-7}. The red color in the images shows the (binarized) differences in the images in between two conditions (CentOS6 and CentOS7).

\hfill \break
\begin{center}
\includegraphics[width=.99\linewidth]{std2str.png}
\captionof{figure}{Illustrates the differences in the standard2std file. \href{https://drive.google.com/file/d/1ffPVXBT2GZ-le3kQzfwvxS_0Z7StnGrk/view?usp=sharing}{Link}}
\caption*{(Subject: 105216; Filename: standard2str.nii.gz; Dice coeff.: 0.2*10\textsuperscript{-7} ; NRMSE: 0.036)}
\label{fig:prefreesurfer_std_file}
\end{center}
\hfill \break

\section{FreeSurfer} \label{sec:Freesurfer}
FreeSurfer results presented in this section does not take into account the files that were found to be different in PreFreeSurfer pipeline. The number of files that has differences and which are common to all the subjects, was found to be 61 files (23 .nii.gz files and 38 .mgz files). Table~\ref{tab:FreeSurfer_Metic_Values} contains the details about the mean, median and standard deviation of NRMSE and Dice coefficient values. Figure~\ref{fig:freesurfer_metric_values} illustrates the NRMSE and Dice coefficient values in the form of mat
\hfill \break
\begin{center}
\begin{tabular}{|l|l|l|}
\hline
\textbf{Item}      & \textbf{Nrmse} & \textbf{Dice} \\ \hline
Mean               & 0.01766    & 0.6953  \\ \hline
Median             & 0.00973    & 0.9262   \\ \hline
Standard Deviation & 0.0181     & 0.4051   \\ \hline
\end{tabular}
\captionof{table}{NRMSE \& DICE values for FreeSurfer processing on CentOS6 and CentOS7}
\label{tab:FreeSurfer_Metic_Values}
\end{center}
\hfill \break

Table \ref{tab:FreeSurfer_Metic_Values} contains the mean, median and standard deviation of files having differences from FreeSurfer preprocessing on CentOS6 and CentOS7. The NRMSE value from the FreeSurfer pipeline has an average of 0.0176 and the Dice Coefficient average is 0.69. Dice Coefficient shows that out of the results we obtained from FreeSurfer processing on two conditions, there was 69\% similarity while taking the average of dice coeffiecient of every file that we found to have a checksum difference.
\vskip 0.2in
Figure \ref{fig:freesurfer_metric_values} contains the matrices showing the plotted values of NRMSE (left) and Dice Coefficient (right). Each line in the matrix corresponds to a file common to all the subjects. Each column represents a subject and each row represents the same file in 5 different subjects. The NRMSE value appears brighter if the values are high and it appears darker if the values are low. The Dice coefficient values appears darker if the images are dissimilar and appears white if the images are similiar.

\begin{center}
\includegraphics[width=.5\linewidth]{freesurfer-nrmse.png}%
\includegraphics[width=.5\linewidth]{freesurfer-dice.png}
\captionof{figure}{FreeSurfer metric values}
\caption*{(i) NRMSE (left) (ii)Dice Coefficient (right)}
\label{fig:freesurfer_metric_values}
\end{center}

Figure~\ref{fig:freesurfer_ribbon_file} illustrates the differences in the ``ribbon.nii.gz" file. The file was taken from subject ``105216". The file illustrated here has an NRMSE value of 0.081 and the Dice coefficient value is 0.993. The red color in the image shows the difference in between the images in two conditions (CentOS6 and CentOS7).
\hfill \break
\begin{center}
\includegraphics[width=\linewidth]{ribbon_screenshot_freesurfer.png}
  \captionof{figure}{Differences in the ribbon file from FreeSurfer Result (Red shaded region shows the differences). \href{https://drive.google.com/file/d/1YHV4PHGzkhgOVE5slMNoDksBMqHSLwjS/view?usp=sharing}{Link}}
\caption*{(Subject: 105216; Filename: ribbon.nii.gz; Dice coeff.: 0.993 ; NRMSE: 0.081)}
\label{fig:freesurfer_ribbon_file}
\end{center}
\hfill \break

Figure~\ref{fig:freesurfer_aseg_file} illustrates the differences in the ``aseg.hires.mgz" file. The file was taken from subject ``105216". The file illustrated here has an NRMSE value of 0.010 and the Dice coefficient value is 0.987. The yellow color in the image shows the difference in the images in between two conditions (CentOS6 and CentOS7).
\hfill \break
\begin{center}
\includegraphics[width=.99\linewidth]{aseg-hires-red.png}
\captionof{figure}{FreeSurfer Aseg hires file (Red shaded region shows the differences in the image in between two conditions). \href{https://drive.google.com/file/d/1WUwWp5muXvotbMQqTqR5LcMyncH1ET3P/view?usp=sharing}{Link}}
\caption*{(Subject: 105216; Filename: aseg.hires.mgz; Dice coeff.: 0.987 ; NRMSE: 0.010)}
\label{fig:freesurfer_aseg_file}
\end{center}
\hfill \break

Figure~\ref{fig:freesurfer_wm_file} illustrates the differences in the ``wm.hires.nii.gz" file. The file was taken from subject ``105216". The file illustrated here has an NRMSE value of 0.074 and the Dice coefficient value is 0.994. The grey color around the yellow shaded parts in the image shows the difference in the images in between two conditions (CentOS6 and CentOS7).

\hfill \break
\begin{center}
\includegraphics[width=.99\linewidth]{wm-red-white.png}
\captionof{figure}{FreeSurfer WM hire file (Red shaded region shows the file containing the difference). \href{https://drive.google.com/file/d/1PzR21tsDOZTqDOcoh1zFyDe7ZXB_LXfJ/view?usp=sharing}{Link}}
\caption*{(Subject: 105216; Filename: wm.hires.nii.gz; Dice coeff.: 0.994 ; NRMSE: 0.074)}
\label{fig:freesurfer_wm_file}
\end{center}
\hfill \break

\section{PostFreeSurfer}\label{sec:Postfreesurfer}
25 (.nii.gz) were found to have inter-OS differences after postfreesurfer processing. These files are created as part of PostFreeSurfer processing alone, and they are common to all the five subjects. Table~\ref{tab:PostFreeSurfer_Metic_Values} contains the mean, median and standard deviation values of the files having differences across all the subjects.
\hfill \break
\begin{center}
\begin{tabular}{|l|l|l|}
\hline
\textbf{Item}      & \textbf{Nrmse} & \textbf{Dice} \\ \hline
Mean               & 0.0347     & 0.6768   \\ \hline
Median             & 0.0386    & 0.9829   \\ \hline
Standard Deviation & 0.0254    & 0.4584   \\ \hline
\end{tabular}
\captionof{table}{NRMSE \& DICE values for PostFreeSurfer processing on CentOS6 and CentOS7}
\label{tab:PostFreeSurfer_Metic_Values}
\end{center}
\hfill \break

Table \ref{tab:PostFreeSurfer_Metic_Values} contains the mean, median and standard deviation values of files having differences from PostFreeSurfer pipeline on CentOS6 and CentOS7 operating systems. The NRMSE value from the PostFreeSurfer pipeline has an average of 0.034 and the Dice Coefficient average is 0.67. Dice Coefficient shows that out of the results we obtained from PostFreeSurfer processing on two conditions, there was 67\% similarity while taking the average of dice coeffiecient of every file that we found to have a checksum difference.
\vskip 0.2in
Figure \ref{fig:postfreesurfer_metric_values} contains the matrices showing the plotted values of NRMSE (left) and Dice Coefficient (right). Each line in the matrix corresponds to a file common to all the subjects. Each column represents a subject and each row represents the same file in 5 different subjects. The NRMSE value appears brighter if the values are high and it appears darker if the values are low. The Dice coefficient values appears darker if the images are dissimiliar and appears bright if the images are similiar.

\hfill \break
\begin{center}
\includegraphics[width=.5\linewidth]{postfreesurfer-nrmse.png}%
\includegraphics[width=.5\linewidth]{postfreesurfer-dice.png}
\captionof{figure}{PostFreeSurfer metric values}
\caption*{(i) NRMSE (left) (ii)Dice Coefficient (right)}
\label{fig:postfreesurfer_metric_values}
\end{center}
\hfill \break

Figure~\ref{fig:postfreesurfer_high_nrmse} illustrates the differences in the ``aparc.a2009s+aseg.nii.gz" file. The file was taken from subject ``101410". The file illustrated here has an NRMSE value of 0.101 and the Dice coefficient value is 0.97. The red colored regions in the image shows the difference in the image in between two conditions (CentOS6 and CentOS7).

\hfill \break
\begin{center}
\includegraphics[width=\linewidth]{postfreesurfer-nrmse-aparc-a2009s+aseg}
\captionof{figure}{Illustrates the differences in the file with the highest NRMSE value in PostFreeSurfer Processing. \href{https://drive.google.com/file/d/1eeofWaPyk-A2pQd6LlieQwbMYVOrfeI3/view?usp=sharing}{Link}}
\caption*{(Subject: 101410; Filename: aparc.a2009s+aseg.nii.gz; Dice coeff.: 0.97 ; NRMSE: 0.101)}
\label{fig:postfreesurfer_high_nrmse}
\end{center}
\hfill \break

Figure~\ref{fig:postfreesurfer_ribbon_file} illustrates the differences in the ``T1w/ribbon.nii.gz" file. The file was taken from subject ``105216". The file illustrated here has an NRMSE value of 0.038 and the Dice coefficient value is 0.995. The red colored regions in the image shows the difference in the image in between two conditions (CentOS6 and CentOS7).

\hfill \break
\begin{center}
\includegraphics[width=\linewidth]{postfreesurfer-ribbon.png}
  \captionof{figure}{Illustrates the differences in the T1w Ribbon file. \href{https://drive.google.com/file/d/1PCj6gtoQSe3Cn6NAWuy2ZJq6xvm0tjh1/view?usp=sharing}{Link}}
\caption*{(Subject: 105216; Filename: T1w/ribbon.nii.gz; Dice coeff.: 0.995 ; NRMSE: 0.038)}
\label{fig:postfreesurfer_ribbon_file}
\end{center}
\hfill \break

Figure~\ref{fig:postfreesurfer_aparc_file} illustrates the differences in the ``MNINonlinear/aparc+aseg.nii.gz" file. The file was taken from subject ``101006". The file illustrated here has an NRMSE value of 0.073 and the Dice coefficient value is 0.98. The red colored regions in the image shows the difference in the image in between two conditions (CentOS 6 and CentOS7).
\hfill \break
\begin{center}
\includegraphics[width=.99\linewidth]{aparc+aseg_post.png}
\captionof{figure}{Illustrates the differences in the segmentation file. \href{https://drive.google.com/file/d/1_ZyAtveS1oVle8tAXeKsp1_4KdS2MVoB/view?usp=sharing}{Link}}
\caption*{(Subject: 101006; Filename: MNINonlinear/aparc+aseg.nii.gz; Dice coeff.: 0.98 ; NRMSE: 0.073)}
\label{fig:postfreesurfer_aparc_file}
\end{center}
\hfill \break

\section{fMRIVolume}\label{sec:fMRI}
1152 files were found to have inter-OS differences after fMRIVolume processing which were common to all subjects. Table~\ref{tab:fMRIVolume_Metric_Values} contains the mean, median and standard deviation values of the files having differences after fMRIVolume processing.
\hfill \break
\begin{center}
\begin{tabular}{|l|l|l|}
\hline
\textbf{Item}      & \textbf{Nrmse}  & \textbf{Dice} \\ \hline
Mean               & 0.0160    & 0.5605   \\ \hline
Median             & 0.0090     & 0.7511   \\ \hline
Standard Deviation & 0.0196     & 0.3769   \\ \hline
\end{tabular}
\captionof{table}{NRMSE \& DICE values for fMRIVolume processing on CentOS6 and CentOS7}
\label{tab:fMRIVolume_Metric_Values}
\end{center}
\hfill \break

Table \ref{tab:fMRIVolume_Metric_Values} contains the mean, median and standard deviation of the files having differences produced by the fMRIVolume pipeline on CentOS6 and CentOS7 operating systems. The NRMSE value from the fMRIVolume pipeline has an average of 0.016 and the Dice Coefficient average is 0.56. Dice Coefficient shows that out of the results we obtained from fMRIVolume processing on two conditions, there was 56\% similarity while taking the average of dice coeffiecient of every file that we found to have a checksum difference.
\vskip 0.2in
Figure \ref{fig:fMRI_metric_values} contains the matrices showing the plotted values of NRMSE (left) and Dice Coefficient (right) with respect to the files having differences from the fMRIVolume pipeline preprocessing. Each line in the matrix corresponds to a file common to all the subjects. Each column represents a subject and each row represents the same file in 5 different subjects. The NRMSE value appears brighter if the values are high and it appears darker if the values are low. The Dice coefficient values appears darker if the images are dissimiliar and appears white if the images are similiar.

\hfill \break
\begin{center}
\includegraphics[width=.5\linewidth]{fmri_NRMSE.png}%
\includegraphics[width=.5\linewidth]{fmri_DICE.png}
\captionof{figure}{fMRIVolume metric values}
\caption*{(i) NRMSE (left) (ii)Dice Coefficient (right)}
\label{fig:fMRI_metric_values}
\end{center}
\hfill \break

Figure~\ref{fig:allgrey_matter} illustrates the differences in the ``AllGreyMatter.nii.gz" file. The file was taken from subject ``101006". The file illustrated here has an NRMSE value of 0.074 and the Dice coefficient value is 0.99. The yellow colored regions in the image shows the difference in the image in between two conditions (CentOS 6 and CentOS7).

\hfill \break
\begin{center}
\includegraphics[width=.75\linewidth]{Frontview2.png}%
\captionof{figure}{All Grey Matter file with differences (Blue shaded region shows the CentOS 6 output and yellow region shows the CentOS 7 output). \href{https://drive.google.com/file/d/1dzpyMalJ6_ox8jLedKvhKrP2ew7C7N-r/view?usp=sharing}{Link}}
\caption*{(Subject: 101006; Filename: AllGreyMatter.nii.gz; Dice coeff.; 0.99; NRMSE; .074)}
\label{fig:allgrey_matter} 
\end{center}
\hfill \break

Figure~\ref{fig:tfMRI_mask_file} illustrates the differences in the ``tfMRI\_MOTOR\_LR\_mc\_mask.nii.gz" file. The image on the left is the output from CentOS6 and the image on the right is the output from CentOS7. The files are taken from subject ``101006". The files illustrated here has an NRMSE value of 0.275 and the Dice coefficient value is 0.002. While comparing the images, we can find the differences visually.

\hfill \break
\begin{center}
\includegraphics[width=.5\linewidth]{tfMRI_MOTOR_LR_mc_mask_centos6.png}%
\includegraphics[width=.5\linewidth]{tfMRI_MOTOR_LR_mc_mask_centos7.png}
\captionof{figure}{Differences in tfMRI Motor MC Mask file. \href{https://drive.google.com/file/d/1_XpZnroleOlJmkvXHZA84iknT5UxtO14/view?usp=sharing}{Link}}
\caption*{(Subject: 105216; Filename: tfMRI\_MOTOR\_LR\_mc\_mask.nii.gz (CentOS6 on left, CentOS7 on right); Dice coeff.; 0.0002; NRMSE; 0.275)}
\label{fig:tfMRI_mask_file}
\end{center}
\hfill \break

Figure~\ref{fig:scout_gdc_file} illustrates the differences in the ``Scout\_gdc\_undistorted2T1w.nii.gz" file. The files are taken from subject ``101410". The files illustrated here has an NRMSE value of 0.010 and the Dice coefficient value is 0.778. The blue shaded region shows the difference in the image in between the two conditions.

\hfill \break
\begin{center}
\includegraphics[width=.99\linewidth]{gdc_scout.png}
\captionof{figure}{Differences in the Scout\_gdc\_undistorted2T1w file. \href{https://drive.google.com/file/d/1a6XuyGDm8pJ2WFccWEuHQnKpN-FL_Ryi/view?usp=sharing}{Link}}
\caption*{(Subject: 101410; Filename: Scout\_gdc\_undistorted2T1w.nii.gz ; Dice coeff.; 0.778; NRMSE; 0.010)}
\label{fig:scout_gdc_file}
\end{center}
\hfill \break

\section{Effect of changing Subject Vs. changing Condition}\label{sec:comparison}
In order to quantify the effect of operating system on the output image and to compare the difference with the anatomical differences between subjects, we made a comparison with two subjects processed in the same condition (101107 and 101006) vs. the same subjects processed in two different conditions (CentOS6 and CentOS7). Subjects 101107 and 101006 processed using PreFreeSurfer on CentOS7 was compared against 101006, 101007 processed on CentOS6 and CentOS7 respectively. Table \ref{tab:comparison_table} contains the mean, median and standard deviation across the said conditions.

\hfill \break
\begin{center}
\begin{tabularx}{.98\textwidth}{|c|c|c|c|c|c|c|}
\hline
Item  & \makecell[l]{NRMSE\\(101006 \\ vs.\\101107)} & \makecell[l]{Dice Coeff.\\(101006\\ vs.\\101107)} & \makecell[l]{NRMSE\\101006\\CentOS\\(6Vs7)} & \makecell[l]{Dice Coeff.\\ 101006\\CentOS\\(6Vs7)} & \makecell[l]{NRMSE\\101107\\CentOS\\(6Vs7)} & \makecell[l]{Dice Coeff.\\ 101107\\CentOS\\(6Vs7)} \\ \hline
Average            & 0.092        & .265      & 0.0066     & 0.3014   & .0077   & .3001     \\ \hline
Median             & 0.078    & .0109       & 0.004          & 0.0217           & .0045     & .021  \\ \hline
\makecell[l]{Std.\\Dev} & 0.073     & 0.377           & 0.0092         & 0.3898   & .0100       & .385 \\ \hline
\end{tabularx}
\captionof{table}{Comparison: Anatomical differences vs. Effect of operating system}
\label{tab:comparison_table}
\end{center}
\vskip .2in
Comparing the metric values obtained from anatomical differences and effect of operating system differences, the operating system differencs quantified by NRMSE value is 1/15\textsuperscript{th} of the differences that are caused by the anatomical variations of the subjects.
