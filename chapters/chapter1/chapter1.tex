\chapter{Introduction}
\section{Reproducibility}
\section{Why is it important}

Reproducibility is a defining feature of science, but the extent to which it characterizes current research is unknown. Scientific claims should not gain credence because of the status or authority of their originator but by the replicability of their supporting evidence \cite{aac4716}.  

The lack of computational reproducibility threatens data science in several domains \cite{Gla15}. Open Science Collaboration \cite{aac4716} conducted replications of 100 experimental and correlational studies pulbished in three psychology journals and using high-powered designs and original materials when available. The results shows that if no bias in original results is assumed, combining original and replication resutls left 70 percentage with statistically significant effects.
 
We focus on the pre-processing of data from the Human Connectome Project (HCP) \cite{Gla13}.
A key component of the scientific method is the ability to revisit and replicate previous results.Registering information about an experiment allows scientists to interpret and
understand results, as well as verifying that the experiment was performed according to acceptable procedures. To carry out a computer science experiment we make use of the entiretyof the computer which is used for running the experiment. Repeating a computer science experiment doesn't require a scientist to rewrite the code from scratch but getting access to the code and resources suffice. Thus in principle a well documented code should be easily reproducible, but that is not the case. Computing environments change rapidly in terms of hardware as well as software. So just documenting the code base won't be adequate enough for easily reproducing the experiments.

Thus the rapid changes occuring in computing environments are complex and might cause differences in outputs eventhough the correct version of the code was ran on the exact same computing environment.To address this problem virtual machines could be used. But the overheads in terms of performance and management(creating, storing and transferring them) can be high and, in some fields of Computer Science such as Computer Systems research it can't be accounted easily. \cite{7092948}

\section{How it makes science more accessible and valid}
https://www.ncbi.nlm.nih.gov/pmc/articles/PMC2981311/
\section{How to make an experiment reproducible}
http://journals.plos.org/ploscompbiol/article?id=10.1371/journal.pcbi.1003285

