\chapter{Introduction}
\section{Reproducibility}
\begin{itemize}
 \item A multi-modal parcellation of human cerebral cortex.
\end{itemize}
\paragraph{
Reproducibility is a defining feature of science, but the extent to which it characterizes current research is unknown. Scientific claims should not gain credence because of the status or authority of their originator but by the replicability of their supporting evidence \cite{aac4716}. The lack of computational reproducibility threatens data science in several domains \cite{Gla15}. Open Science Collaboration \cite{aac4716} conducted replications of 100 experimental and correlational studies pulbished in three psychology journals and using high-powered designs and original materials when available. The results shows that if no bias in original results is assumed, combining original and replication resutls left 68 percentage with statistically significant effects.
We aim at,
\begin{enumerate} 
  \item  Quantifying the effect of operating system version updates on biological results and 
  \item  Identifying the software tools in a pipeline that are responsible for this effect.
\end{enumerate} 
We focus on the pre-processing of data from the Human Connectome Project (HCP) \cite{Gla13}.
}
