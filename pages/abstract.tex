\begin{abstract}
The lack of computational reproducibility threatens data science in several domains\footnote{\url{www.nature.com/news/1-500-scientists-lift-the-lid-on-reproducibility-1.19970}}. This study focus on neuroimaging pipelines and aims to (i) identify the effect of operating system on neuroimaging pipelines (ii) quantify the effect with the use of standard metrics. %(iii) verification of these differences to find out if they are visually substantial.

To conduct this study, a framework for evaluating the reproducibility of these neuroimaging pipelines was developed. The two major functionalities of this framework being, processing of the subjects and the analysis of the processed data. The framework consists of Docker containers, CBRAIN\footnote{\url{https://mcin-cnim.ca/technology/cbrain/}}, Boutiques\footnote{\url{http://boutiques.github.io/}} descriptors, Reprozip\footnote{\url{https://www.reprozip.org/}} and Repro-tools\footnote{\url{https://github.com/big-data-lab-team/repro-tools}}.

  Docker images were created from CentOS versions which contained the HCP preprocessing pipelines. Boutiques descriptors were used for deploying these Docker containers to CBRAIN (a web-platform to manage computing resources and data). Provenance information of these pipelines were recorded using Reprozip. The identification and quantification of the differences were done using the Reprotools. Normalized root mean square error and dice coefficient of similarity were the metrics used for quantifying the differences. 

Two kinds of differences can occur in the output images, which are, inter-OS and inter-run differences. Inter-OS differences are differences that are created due to the changes in the operating system version. Inter-run differences are differences that occur with in the same condition due to reasons like pseudo-random process, silent crashes etc.

With the help of the framework that was developed, pre-processing of data from the Human Connectome Project (HCP) was done. PreFreeSurfer, FreeSurfer, PostFreeSurfer and fMRIVolume are the preprocessing pipelines used for conducting this study. Raw data of 5 subjects from the HCP S500 release are used for the study.

Inter-OS differences were identified in the output images from the HCP preprocessing pipelines between conditions (CentOS6 and CentOS7). With the help of Repro-tools, normalized root mean square error and dice coefficient value of these differences was computed. Normalized root mean sqaure error ranges from 0 to 0.27 and dice coefficient value ranges from 0 to 1. No inter-run differences were identified.

No inter-run differences shows that the pipelines are not using pseudo-random processes and there are no unreported crashes. The inter-OS differences acorss the pipelines identified are visually substantial. There is a high chance that these inter-OS differences are coming from the numerical instability of these HCP preprocessing pipelines.

With the use of containers and sticking to the same condition for conducting the experiments, the variations in these differences can be masked. But the better solution to this problem is to fix the numerical instability of HCP preprocessing pipelines.
\end{abstract}
