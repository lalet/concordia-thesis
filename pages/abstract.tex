\begin{abstract}
The lack of computational reproducibility threatens data science in 
several domains. In particular, it has been shown that different 
operating systems can lead to different analysis results. This study 
identifies and quantifies the 
effect of the operating system on neuroimaging analysis pipelines. 
%framework
We developed a framework to evaluate the reproducibility of these 
neuroimaging pipelines across operating systems. The framework 
themselves leverages software containerization and system-call 
interception to record results provenance without having to instrument 
the pipelines. A tool (Repro-tools) compares results obtained under different conditions.
% Experiment description
We used our framework to evaluate the effect of the
operating system on results produced by pipelines from the Human
Connectome Project (HCP), a large open-data initiative to study the human brain.
In particular, we focused on pre-processing pipelines for
anatomical and functional data, namely PreFreeSurfer, FreeSurfer,
PostFreeSurfer, and fMRIVolume.
We used data from five subjects released by
the HCP.
% Results
Results highlight substantial differences in the output of the HCP
pipelines obtained in two versions of Linux (CentOS6 and CentOS7).
Inter-OS differences corresponding to normalized root mean square
errors of up to 0.27 were observed, which corresponds to visually
important differences. We provide visualizations of the most important
differences for various pipeline steps. No meaningful inter-run
differences were observed, which shows that the inter-OS differences do
not originate from the use of pseudo-random numbers or silent crashes
of the pipelines.
% Discussion
We hypothesize that the observed inter-OS differences come from
numerical instabilities in the pipelines, triggered by rounding and
truncation differences that originate in the update of mathematical
libraries in different systems. An apparent solution to this issue
is to freeze the execution environment using, for example, software containers.
However, this would only mask instabilities while they should ultimately be
corrected in the pipelines.
\end{abstract}
