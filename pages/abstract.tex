\begin{abstract}
The lack of computational reproducibility threatens data science in several domains~\cite{Gla15,10.1371/journal.pone.0038234}. We aim at (1) quantifying the effect of operating system version updates on biological results and (2) identifying the software tools in a pipeline that are responsible for this effect. We focus on the pre-processing of data from the Human Connectome Project (HCP)~\cite{Gla13}. PreFreeSurfer, FreeSurfer, PostFreeSurfer and fMRIVolume are the preprocessing pipelines we used for conducting our study. We used raw data of 5 subjects (unprocessed neuroimaging data) from the HCP S500 Release.

A framework for evaluating the reproducibility of these neuroimaging pipelines was developed which consists of two major components. The processing of the subjects and the analysis of these preprocessed data. The framework consists of Docker containers, CBRAIN\footnote{\url{https://mcin-cnim.ca/technology/cbrain/}}, Boutiques\footnote{\url{http://boutiques.github.io/}} descriptors, Reprozip\footnote{\url{https://www.reprozip.org/}} and Repro-tools\footnote{\url{https://github.com/big-data-lab-team/repro-tools}}.

Docker images were created from CentOS versions 6.8 and 7.2, which contained the HCP preprocessing pipelines. We also created wrapper script around the preprocessing pipelines to prevent file corruption, record the hardware and software library version details. These containers were deployed on a server using CBRAIN (a web-platform to manage computing resources and data). Boutiques descriptors were used deploying the Docker containers to CBRAIN. Provenance information of these pipelines were also recorded using Reprozip. The identification and quantification of the differences were done using the Reprotools. Normalized root mean square error and dice coefficient of similarity are the metrics we used for quantifying the differences.

For each subject, preprocessing was repeated two times on both versions of CentOS to check for inter-OS and inter-run differences. Inter-OS differences are differences that are created due to the changes in the operating system version. Inter-run differences are differences that occur with in the same condition due to reasons like pseudo-random process, silent crashes etc.

We could identify inter-OS differences in the output images between conditions (CentOS6 and CentOS7). The inter-OS differences we have identified are substantial and thus HCP preprocessing pipelines are affected by the underlying operating system. With the help of Repro-tools we were able to quantify the differences. Normalized root mean sqaure error ranges from 0 to 0.27 and dice coefficient value ranges from 0 to 1. No inter-run differences were identified.
\end{abstract}
